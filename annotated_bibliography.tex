<<<<<<< HEAD
\documentclass[12pt]{article}
\usepackage{setspace}
\singlespace
\usepackage[left=1in,right=1in,top=1in,bottom=1in]{geometry}
\usepackage[T1]{fontenc}
\usepackage[utf8]{inputenc}
\usepackage{hyperref}

\title{\textbf{Annotated Bibliography: Accessibility UX/UI Design}}
\author{Noelynn Faith Batalingaya}
\date{September 25, 2025}

\begin{document}
\maketitle

% Show ONLY the 5 new sources:
\nocite{alshayban2020androidaccess,duan2024mockupfeedback,duan2024uicrit,muniz2024figmatemplates,zhang2021screenrecognition}

\bibliographystyle{plain-annote}
\bibliography{bibliography}

\end{document}
=======
<<<<<<< HEAD
% This is a template for your annotated bibliography. The only
% things you need to edit in this file are the title and author. You'll
% write the annotations themselves in the "annote" fields of the entries in
% bibliography.bib.
%
% To compiling documents on the command line, you'll run the following:
% latexmk -pdf annotated_bibliography.tex
%
% If you there is an error in the .bib file, you might need to delete the
% intermediate .aux, .bbl, and .blg files that are generated during the
% compilation process.


% Using this makes it so that all the references show up regardless of whether
% or not they were actually cited in the rest of the document. Since we are
% only displaying the bibliography we won't be citing anything anywhere, so we
% need this.


\documentclass[12pt]{article}
\usepackage{setspace}
\singlespace
\usepackage[left=1in,right=1in,top=1in,bottom=1in]{geometry}
\usepackage{hyperref}

\title{\textbf{Annotated Bibliography: Accessibility UX/UI Design}}
\author{Noelynn Faith Batalingaya}
\date{September 25, 2025}

\begin{document}
\maketitle

% Show all entries from bibliography.bib, even if not cited
\nocite{*}

\bibliographystyle{plain-annote}
\bibliography{bibliography}

\end{document}
>>>>>>> f96c51ae2acaa5936b4f28f0efde8aed3160b65f
