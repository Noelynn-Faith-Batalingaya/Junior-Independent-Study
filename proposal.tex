% Project Topic Proposal — Shift-Left Accessibility 
\documentclass[12pt]{article}
\usepackage[left=1in,right=1in,top=1in,bottom=1in]{geometry}
\usepackage{setspace}
\usepackage{graphicx}
\singlespace

\title{\textbf{Shift-Left Accessibility for UX/UI: A Figma Preflight vs.\ Code-Time Checks (Pilot)}}
\author{Noelynn Faith Batalingaya}
\date{September 2025}

\begin{document}
\maketitle

\section*{Problem Statement \& Motivation}
My project starts from a simple, practical question: \emph{How much accessibility can we catch while the interface is still a Figma mockup, and does doing that early save time once we write code?} In real teams, accessibility often shows up late and creates rework. I want to test whether a short, repeatable "preflight" checklist during design reviews prevents common issues from ever reaching implementation.

\section*{Background \& Prior Work}
Accessibility is ultimately about real people being able to use what we ship—from screen reader and keyboard-only users to people with low vision, older adults, and anyone using a cracked phone outdoors. HCI research argues that inclusive design should start at the beginning, not as a last-minute checklist \cite{bennett2018inclusive}. Recent studies show that many problems are visible in the mockups themselves—low color contrast, small hit targets, unclear component states, and weak hierarchy—and can be assessed systematically in Figma before any code exists \cite{huang2024a11yfigma, chen2024figmaapps}. Work with UX practitioners also highlights real barriers (time, ownership, uneven training) and recommends lightweight checklists and design-system constraints as realistic, adoptable improvements \cite{shi2023uxaccesspractice}. My pilot turns these insights into a small but measurable study.

\section*{Project Scope \& Goals}
\textbf{Goal 1.} Create a one-page \emph{Figma Accessibility Preflight} centered on five checks: (a) contrast \& legibility, (b) hierarchy \& structure, (c) target size \& spacing, (d) state visibility (focus/hover/active/disabled/error), and (e) basic semantics/affordances (labels, recognizable controls).\\
\textbf{Goal 2.} Implement two matching screens (Home + Register) as a tiny website and run code-time checks (keyboard, screen reader, automated tools).\\
\textbf{Goal 3.} Compare \emph{which issues are caught when} and the \emph{minutes to fix} at design vs.\ code time.

\textbf{Research Questions.} 
\textbf{RQ1:} Which issue types are reliably caught at design time versus only at code time (e.g., focus order, announcements)? 
\textbf{RQ2:} Does the preflight reduce “minutes to fix” compared to fixing after implementation?

\section*{Planned Software (what I will build)}
To keep effort manageable and the focus on accessibility (not tools), I will build a minimal two-page site using HTML/CSS plus a few lines of JavaScript:
\begin{itemize}
  \item \textbf{Home:} semantic landmarks (\texttt{header/nav/main/footer}), a visible \emph{Skip to main content} link, proper headings, high-contrast tokens, and a clear \texttt{:focus-visible} outline.
  \item \textbf{Register:} explicit \texttt{<label for>} on each field, hint text via \texttt{aria-describedby}, inline error messages with role="alert", and focus moves to the first error on submit. All key interactions should work by keyboard (Tab/Shift-Tab/Enter/Space/Esc).
\end{itemize}

\section*{Method \& Measures}
\textbf{Phase A — Design-time Preflight.} I will run the five checks on two Figma mockups (Home, Register). For every issue, I will log: \texttt{screen\_id, phase=design, category, severity, found\_by=pre-flight, fix\_minutes, notes}. I'll then update the mockups and record the design fix minutes.

\textbf{Phase B - Code-time Checks.} I will build the two pages and perform three passes:
\begin{enumerate}
  \item {Keyboard pass:} reach everything with Tab/Shift-Tab, visible focus, no traps, expected Enter/Space/Esc behavior.
  \item {Screen reader pass (macOS VoiceOver):} names/roles/labels, headings/landmarks, and whether errors are announced.
  \item {Automated audits:} Lighthouse and axe DevTools for quick, repeatable checks.
\end{enumerate}
Each issue will be logged with \texttt{phase=code} and code fix minutes. 

\textbf{Analysis.} I will compare counts by phase and category and graph the average \emph{minutes to fix} at design vs.\ code. Based on prior work, I expect the preflight to catch many visual/structural issues quickly while code-time checks remain essential for focus order and screen reader announcements \cite{huang2024a11yfigma, chen2024figmaapps, shi2023uxaccesspractice}. I will discuss limits (small pilot, student context) and outline how to expand next semester (more screens; possibly recruit assistive technology users; compare teams with vs.\ without the preflight).

\section*{Illustration (logic only)}
This proposal includes one figure: a logic diagram summarizing the flow from the Figma preflight to code-time checks and the final comparison of issues and fix minutes (Figure~\ref{fig:logic}).

\begin{figure}[h]
  \centering
 \includegraphics[width=0.95\linewidth]{imgs/logic-diagram.png}
  \caption{Workflow: Preflight on mockups $\rightarrow$ Implement pages $\rightarrow$ Keyboard/Screen Reader/axe checks $\rightarrow$ Compare issues \& fix minutes.}
  \label{fig:logic}
\end{figure}

\section*{Expected Contributions}
Deliverables will include: (1) a short, printable preflight that can be used in design critique; (2) a tiny set of accessible component patterns (focus ring, skip link, labeled inputs, reduced motion); and (3) a small dataset that shows where early checks save time and where code-time checks are still required. These are practical outcomes for student teams and a solid foundation to scale into a larger Senior IS.

\newpage
\section*{Appendix: Feature List (implementation order)}
\begin{enumerate}
  \item Add semantic landmarks and a \emph{Skip to main content} link.
  \item Set heading hierarchy (H1/H2/H3) and spacing scale.
  \item Add a clear \texttt{:focus-visible} outline (high contrast, offset).
  \item Define color tokens and verify contrast (text \& UI) against WCAG AA.
  \item Build \textbf{Home} navigation: keyboard-reachable links; hover/active states; visible focus.
  \item Build \textbf{Register} form with \texttt{<label for>} and hint text via \texttt{aria-describedby}.
  \item Inline error message with \texttt{role="alert"} and focus-to-first-error on submit.
  \item Enforce \textbf{target size} minimums ($\sim 44\times 44$ px) and adequate spacing between controls.
  \item Respect \texttt{@media (prefers-reduced-motion)} for users who limit animation.
  \item Run Lighthouse/axe; log issues + \emph{fix\_minutes}; capture before/after screenshots.
  \item[\textit{Stretch}] Add a basic \textbf{table pattern} (caption, header cells with scope) + one-sentence text summary.
  \item[\textit{Stretch}] Add a short \textbf{alt-text guide} and test a tiny image gallery (good vs.\ poor examples).
  \item[\textit{Stretch}] Optional automation: run Pa11y or jest-axe locally for quick regression checks.
\end{enumerate}

\clearpage
\bibliographystyle{acm}
\bibliography{bibliography}

\end{document}
