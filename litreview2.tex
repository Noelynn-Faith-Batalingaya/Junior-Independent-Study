\documentclass[12pt]{article}

\usepackage[T1]{fontenc}
\usepackage[utf8]{inputenc}
\usepackage{lmodern}
\usepackage[margin=1in]{geometry}
\usepackage{setspace}
\setstretch{1.5}

\usepackage{graphicx}
\usepackage{float}
\graphicspath{{imgs/}}

\usepackage{titlesec}
\usepackage{caption}
\usepackage[hidelinks]{hyperref}

\title{Shift-Left Accessibility for UserExperience/UserInterface: A Figma Preflight\\
versus Code-Time Checks}
\author{Noelynn Faith Batalingaya}
\date{\today}

\begin{document}
\maketitle

\begin{abstract}
Digital accessibility is both a technical and moral imperative. Despite decades of progress in web standards and assistive technologies, many interfaces continue to exclude users with disabilities—often because accessibility is treated as an afterthought rather than a design foundation. This paper explores the emerging concept of \textit{shift-left accessibility}, which moves accessibility checks from code-time to design-time. It presents a Figma plugin prototype that performs accessibility “preflight” checks on color contrast, alternative text, focus indicators, and target size, thereby integrating inclusive design principles into the creative process itself. The study examines how early feedback can reduce rework, promote empathy, and teach accessibility literacy among student designers. Ultimately, this work argues that accessibility is not a restriction but a creative discipline that fosters equity and innovation across digital spaces.
\end{abstract}

\textbf{Keywords:} Accessibility, Inclusive Design, UX/UI, Figma, WCAG, Shift-Left Testing, Human-Centered Design, Assistive Technology

\section{Introduction}
In an increasingly digital world, accessibility has become a defining measure of equity. The interfaces that structure everyday life—classroom portals, job applications, telehealth systems, banking apps—are gateways to opportunity. When those systems exclude users with disabilities, they reinforce historical inequities under the guise of innovation. Accessibility is therefore not a bonus feature; it is an ethical baseline for participation in modern life.

The problem persists partly because accessibility is too often considered late in development. Designers may unknowingly select color combinations that fail WCAG contrast ratios, omit text alternatives for icons, or create navigation sequences that confuse screen readers. Once those design decisions move downstream into production, fixing them requires rewriting code and restructuring layouts, consuming more time and resources than addressing them earlier. These oversights create what some researchers describe as an “accessibility debt”—an accumulation of design flaws that compound across project stages.

Studies continue to show that digital platforms, even those developed by large organizations, regularly fail basic accessibility audits \cite{alshayban2020androidaccess}. The issue is systemic, not simply individual error. Teams face organizational barriers: limited training, unclear ownership, and tight timelines that discourage accessibility testing \cite{shi2023uxaccesspractice}. Consequently, accessibility becomes reactive rather than proactive—something verified after deployment, not designed from the start.

This project adopts a \textit{shift-left} perspective. Borrowed from software testing and DevOps(Development and IT Operations), the shift-left principle encourages identifying and fixing issues earlier in the lifecycle when changes are cheapest and fastest. Applied to accessibility, it suggests that designers themselves can prevent many barriers by checking inclusivity before development begins. This paper introduces the A11y Preflight plugin for Figma, a tool that brings accessibility evaluation into the design phase. Its purpose is twofold: (1) to automate common WCAG checks such as contrast, alternative text, and target size, and (2) to teach designers how their creative decisions affect usability for real people. The central research question asks: \textit{Can early accessibility feedback within design tools reduce downstream issues and foster a culture of inclusive design?}

\section{Background}
Accessibility refers to the design of products and environments that can be used by as many people as possible, including those with disabilities. It bridges human diversity—visual, auditory, cognitive, motor—and technological design. The Web Content Accessibility Guidelines (WCAG) organize accessibility into four core principles: perceivable, operable, understandable, and robust \cite{wcag2023}. Together, these principles define what it means for content to be accessible, usable, and sustainable.

\textbf{Perceivable} means users can see or hear content through alternative forms like captions, transcripts, or sufficient color contrast. \textbf{Operable} ensures users can interact using multiple inputs, such as keyboards, voice, or switches. \textbf{Understandable} means the content and interface are predictable and logically structured, reducing cognitive load. Finally, \textbf{Robust} ensures compatibility with assistive technologies across devices and browsers. These guidelines are not abstract ideals—they directly impact whether a person can register for school, apply for work, or read a medical record independently.

WCAG’s conformance levels (A, AA, AAA) represent increasing stringency. While Level~A ensures basic access, Level~AA improves usability for most users, and Level~AAA represents the gold standard. However, true inclusivity often demands cultural and procedural change beyond compliance. As disability scholars argue, accessibility is not just about “meeting requirements” but about rethinking how technology distributes power and participation.

Timing is equally critical. In many workflows, accessibility evaluation occurs at the very end of production. By then, design artifacts are frozen, and budgets are strained. A \textit{shift-left} approach repositions accessibility as a creative responsibility rather than a corrective measure. It aligns with agile and DevOps philosophies where testing happens iteratively, not sequentially. In this sense, accessibility testing is not simply an audit—it is an act of care embedded in design thinking.

\subsection*{Accessibility in Figma and the Rise of Early Tools}
Figma’s collaborative, web-based nature makes it ideal for early accessibility checks. Plugins can analyze static frames, evaluate color contrast, detect unlabeled elements, and display guidance without interrupting workflow. Previous research confirms that design-time scanning can identify up to 60\% of recurring accessibility issues before implementation \cite{huang2024a11yfigma, chen2024figmaapps}. Yet, while existing tools such as Stark and Able cover contrast and alt text, few integrate broader checks such as focus order or touch target sizing. This gap motivates the A11y Preflight plugin, which aims to consolidate these checks while providing educational feedback that helps designers internalize accessibility standards.

\subsection*{Inclusive Design and User Empathy}
Inclusive design expands accessibility by viewing human diversity as a resource for creativity. Rather than targeting an “average” user, inclusive design anticipates differences in ability, environment, culture, and language. Empathy drives this approach. By engaging users with disabilities in the testing process, designers gain experiential knowledge that enriches their design decisions. This mindset turns accessibility from a checklist into a design ethos, where inclusion shapes aesthetics, structure, and emotion.  

Inclusive design also resonates with social justice principles. Scholars in human-computer interaction argue that technology should amplify, not constrain, participation. Designing for accessibility means designing for flexibility, choice, and dignity—values that ultimately benefit all users. For instance, captions assist not only Deaf users but also people in noisy environments; larger buttons help those with limited mobility and those using small screens. Accessibility, then, enhances usability universally.

\section{Examples of Good and Bad Design}
Design communicates meaning visually. Accessibility is not separate from aesthetics; it is part of it. Figure~\ref{fig:goodbad} illustrates this relationship. The left interface appears visually minimalist but violates core WCAG principles: gray-on-white text lacks contrast, focus states are invisible, and tap targets are small. The right interface achieves visual harmony without sacrificing usability—it increases contrast, enlarges buttons, and adds clear keyboard focus indicators.  

\begin{figure}[H]
  \centering
  \includegraphics[width=\textwidth]{good_bad_example.png}
  \caption{Comparison of inaccessible (left) vs. accessible (right) design components. The right variant preserves visual hierarchy while satisfying WCAG~AA color and interaction standards.}
  \label{fig:goodbad}
\end{figure}

Such contrasts reveal a misconception: accessibility supposedly limits creativity. In reality, constraints foster innovation. By designing within accessibility standards, creators discover more thoughtful uses of color, type, and motion. Moreover, accessible design tends to improve comprehension and conversion for all users. A button that meets color-contrast thresholds is easier for everyone to read; a clear focus outline benefits power users as well as keyboard navigators. Accessibility is therefore not only ethical but also pragmatic—it makes design stronger and more resilient.

\section{Before and After Screenshots}
The transformation from inaccessible to accessible design can be subtle yet powerful. Figure~\ref{fig:before} shows a layout before running the A11y Preflight plugin. Text appears faint, buttons are small, and icons lack alternative descriptions. After applying plugin feedback, the corrected interface (Figure~\ref{fig:after}) increases color contrast to meet WCAG~2.2~AA standards, enlarges touch areas, and adds text labels.  

\begin{figure}[H]
  \centering
  \includegraphics[width=0.9\textwidth]{before.jpg}
  \caption{Before: inaccessible layout with weak contrast, small targets, and missing alternative text.}
  \label{fig:before}
\end{figure}

\begin{figure}[H]
  \centering
  \includegraphics[width=0.9\textwidth]{after.jpg}
  \caption{After: accessibility improved following plugin feedback; contrast and labeling meet WCAG~AA ratios.}
  \label{fig:after}
\end{figure}

While these adjustments might seem cosmetic, they significantly affect how users perceive and interact with a design. An accessible interface communicates attentiveness and respect—it signals that the creator has considered different ways of seeing and navigating the world.

\section{Research Goals and Planned Contribution}
This study investigates whether embedding accessibility checks into design software improves both design quality and designer awareness. It contributes a functional prototype (the A11y Preflight plugin) and an empirical framework for evaluating its impact.

\subsection*{Methodology}
The central research question asks: \textit{Do early, design-stage accessibility checks reduce the number and severity of issues that persist into code compared to conventional, code-time auditing?}  

\textbf{Study Design:} A within-subjects experiment will involve participants designing two short UI mockups—one with the A11y Preflight plugin active and one without. Task order will be counterbalanced to prevent learning effects.  

\textbf{Measures:} (1) number of violations detected by external audit, (2) time to fix each violation, and (3) participants’ reported confidence in identifying accessibility issues. Qualitative interviews will capture reflections on the plugin’s usability and educational value.  

\textbf{Expected Outcome:} It is hypothesized that designs completed with preflight feedback will exhibit fewer downstream violations and shorter correction times. Additionally, participants may demonstrate increased understanding of accessibility concepts, suggesting a dual function for the plugin as both tool and teacher.

\subsection*{Implementation Overview}
The plugin performs the following checks:
\begin{itemize}
  \item \textbf{Contrast Analysis:} Evaluates text-background pairs against WCAG~AA ratios and proposes compliant alternatives.
  \item \textbf{Alt-Text Detection:} Flags images and icons without accessible names.
  \item \textbf{Focus Visibility:} Identifies components lacking visible focus indicators.
  \item \textbf{Target Sizing:} Warns when clickable areas are too small for standard interaction.
  \item \textbf{Color Dependence:} Detects when color alone conveys meaning (e.g., red = error).
\end{itemize}

A new feature added toward the end of development was a simple auto-fix option. Earlier versions of the plugin could only detect issues, but now designers can click ``Fix'' to automatically correct common problems like low contrast, small text, or spacing. This update makes improvements faster and easier, and it also helps designers learn accessibility practices while they work.

Unlike purely diagnostic tools, A11y Preflight is pedagogical. Each issue includes contextual explanations drawn from WCAG principles, turning accessibility evaluation into an active learning experience. For students and early-career designers, this is critical—understanding the ``why'' behind a rule helps transform compliance into empathy.

\section{Related Work}
Previous research shows that most accessibility frameworks—Lighthouse, axe-core, and WAVE—operate at the code level, catching structural violations such as missing ARIA(Accessible Rich Internet Applications) roles or improper headings. While powerful, these tools intervene too late. Design-stage tools remain limited in scope. Stark checks contrast; Able supports alternative text for Figma layers \cite{huang2024a11yfigma, chen2024figmaapps}. Yet few integrate multiple WCAG categories into a single workflow.

Organizational studies reveal that accessibility challenges often arise from unclear team roles and insufficient education \cite{shi2023uxaccesspractice}. Tools that provide contextual feedback within existing design environments can reduce that friction, making accessibility part of normal creative practice rather than a separate audit process.  

Emerging research in AI-driven critique systems offers promise but also caution. Machine learning models can assess screenshots for contrast and layout symmetry but often miss semantic meaning and context \cite{duan2024mockupfeedback, duan2024uicrit}. Human judgment remains essential. Therefore, a balanced approach combining automation and designer awareness remains the most effective strategy.  

Finally, accessibility research increasingly emphasizes emotional and social dimensions. Design that acknowledges difference not only meets standards but also cultivates empathy and belonging. As Ginosar et al.\ argue, human-centered design must move beyond compliance to address how people feel when using technology \cite{ginosar2018parametric}. The A11y Preflight plugin aims to embody that ethos by embedding reflection directly into everyday tools.

\section{Conclusion}
Accessibility cannot wait until the end of production. When it is ignored early, the result is exclusion coded into our most common technologies. This project reimagines accessibility as an integral part of creativity rather than an external audit. By shifting left—bringing accessibility checks into design itself—teams can reduce costs, prevent barriers, and cultivate empathy at the earliest stage of making.

The A11y Preflight plugin is more than a technical artifact; it is a pedagogical bridge between intention and implementation. It empowers designers to recognize that every color, label, and button affects someone’s experience of inclusion. Through automation and education, the plugin embodies a broader truth: accessibility is not about disability—it is about designing a digital world that welcomes everyone. The future of UX/UI depends not only on aesthetics but on ethics—and shift-left accessibility marks a meaningful step toward that future.

\clearpage
\phantomsection
\addcontentsline{toc}{section}{References}
\bibliographystyle{ieeetr}
\bibliography{bibliography}

\end{document}
